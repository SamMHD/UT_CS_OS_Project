\hypertarget{simple-preemptive-sfj-scheduling-simulation}{%
\section{🏷️ Simple Preemptive SFJ Scheduling⏰
Simulation🧪}\label{simple-preemptive-sfj-scheduling-simulation}}

📖 \textbf{\emph{Operating System Concepts}}

✍️ \emph{Saman Mahdanian - 610399197}

\hypertarget{introduction}{%
\subsection{🌱 Introduction}\label{introduction}}

The given code is an implementation of a scheduling algorithm called
\textbf{Shortest Burst Time First (SBTF)}. The purpose of this algorithm
is to schedule the execution of processes in a way that minimizes the
average waiting time of processes. This is achieved by prioritizing the
processes with the shortest burst time. Also, because of the limitations
of the simulation procedure here we use a time quantom of 50ms.

The code also includes a logging feature that records the start and end
times of each process as well as any preemption that occurs.

\hypertarget{implementation}{%
\subsection{🪴 Implementation}\label{implementation}}

The code uses C++17 thread library to implement the scheduling algorithm
in a separate thread. The main thread is responsible for taking user
input and inserting new processes into the scheduler's queue. The
scheduler thread continuously runs and handles the execution of the
processes.

The scheduler thread uses a \texttt{std::set} container to store the
processes in the queue. The set is sorted by the burst time of the
processes, with the process having the shortest burst time at the front.
The scheduler thread takes the front process from the set, starts its
execution and logs the start time. The process's burst time is
decremented by a quantum time (50ms) and if it completes execution, the
process is terminated and removed from the set. If the process still has
burst time remaining, it is inserted back into the set.

The code also uses a \texttt{std::unordered\_map} container to store the
remaining burst time of each process. This is used to update the burst
time of a process if it is preempted.

A std::mutex is used to synchronize access to the shared data structures
between the main thread and the scheduler thread.

\hypertarget{implementation-notes}{%
\subsubsection{🏰 Implementation Notes:}\label{implementation-notes}}

\begin{itemize}
\item
  The program is written in C++ and makes use of several C++ standard
  library features such as the \texttt{iostream}, \texttt{set},
  \texttt{chrono}, \texttt{thread}, \texttt{fstream}, \texttt{mutex},
  and \texttt{unordered\_map} libraries.
\item
  The Process struct is used to represent a process. It contains the
  process ID (pid), the burst time (bt) and the arrival time (art) of
  the process. The \texttt{\textless{}\ operator} is overloaded for this
  struct to compare the burst time of two processes. This is used by the
  set data structure to sort the processes in the ready queue.
\item
  The \texttt{set\textless{}Process\textgreater{}} processes is used to
  store the processes in the ready queue. The set is sorted by the burst
  time of each process, with the process having the shortest burst time
  at the front.
\item
  The \texttt{unordered\_map\textless{}int,\ int\textgreater{}}
  remaining\_time is used to store the remaining burst time of a process
  that was preempted.
\item
  The \texttt{active\_process\_pid} variable is used to keep track of
  the currently active process.
\item
  The \texttt{mtx} variable is used to implement a mutex lock for
  synchronizing access to the shared data structures.
\item
  The \texttt{user\_alive} variable is used to keep track of whether the
  user is still entering processes or not.
\item
  The \texttt{user\_alive\_mtx} and scheduler\_alive\_mtx variables are
  used to implement mutex locks for synchronizing access to the
  \texttt{user\_alive} variable.
\item
  The \texttt{schedule()} function is the main function that is run by
  the scheduler thread. It runs in an infinite loop and repeatedly
  selects the process at the front of the set, starts executing it, and
  then preempts it if a process with a shorter burst time arrives. The
  function also logs all the process scheduling events, such as process
  start, process termination, and process preemption, in a file called
  scheduler.log.
\item
  The \texttt{main()} function is responsible for creating the scheduler
  thread and providing a simple user interface for entering new
  processes. It runs in an infinite loop, where the user is prompted to
  enter the process ID and burst time of a new process. If the user
  enters 0 0, the loop exits, and the program ends.
\end{itemize}

\hypertarget{usage}{%
\subsection{🚀 Usage}\label{usage}}

To run the code, the user must enter the process ID and burst time (in
seconds) of each process. The user can stop entering processes by
entering ``0 0'' as the process ID and burst time. The scheduler thread
will continue to execute the existing processes and terminate when all
processes have completed execution.

The scheduler thread will also log the start and end times of each
process as well as any preemption that occurs in a file named
``scheduler.log''.

The code can be modified to change the quantum time or the scheduling
algorithm as per the requirement.

\hypertarget{conclusion}{%
\subsection{🌳 Conclusion}\label{conclusion}}

The given code is an implementation of the Shortest Burst Time First
(SBTF) scheduling algorithm using C++17 thread library. The code is
designed to minimize the average waiting time of processes and includes
logging feature to record the start and end times of each process and
preemption. The code is well-commented and can be easily understood and
modified as per the requirement.
